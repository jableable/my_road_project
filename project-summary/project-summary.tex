\documentclass{article}
\usepackage{amsthm}
\usepackage{geometry}
\geometry{
	a4paper,
	total={170mm,257mm},
	left=20mm,
	top=20mm,
}



\newtheorem*{problem}{Problem}





\begin{document}
	

		\title{Detecting Road Overpasses from Satellite Images}
		\author{Jared Able}
		\date{}
\maketitle
		

	
	\section{Dataset and Problem}
	
	Our dataset will consist of 10,000+ satellite images of road networks across the world obtained via Google Maps Static API. For a given satellite image with a road network, its label will be the number of times that the network crosses over itself in the form of an overpass. The number of overpasses is one way to measure the complexity of a road network. We will attach these labels to the satellite images via OpenStreetMap data. 
	
	Given an overhead satellite image of some road network, we would like to solve the following problems.
	
	\begin{problem}
			Our primary problem is successfully predicting whether or not an overpass is present within a satellite image. Our secondary problems include:
			\begin{itemize}
				\item getting an accurate count of overpasses within such an image;
				\item extracting and simplifying the graph structure underlying the pictured road network;
				\item and correlating the number of network crossings with geographic and population data such as population size and natural obstacles.
			\end{itemize}
		\end{problem}
		
		

	
	\section{Stakeholders}
	
	Roads and bridges are essential to civilian, commercial, and government transport across the world. As the builder of most roads, the government would be our primary stakeholder. Reflecting their necessity, highways and roads composed 6\% of state and local US government spending in 2020 (about \$200 B). 
	
	
	Our project can both improve old roads and help inform the design of new roads. In the case of old roads, we could simplify the road network by, for example, removing unnecessary loops. In the case of new roads, by considering external factors such as population and terrain, we could predict how many crossings (``how much complexity") would be necessary to design a new road network. Knowing the required complexity for building a road network would help guide planners to a sufficiently complex solution.
	
	Our crossing count will help achieve the following performance indicators:
	
	\begin{itemize}
		\item Reduce cost by simplifying design of old roads
		\begin{itemize}
			\item Simpler roads improve traffic flow and reduce shipping costs
			\end{itemize}
		\item Reduce design time of new roads
			\begin{itemize}
				\item Knowing minimum number of crossings in advance would yield quicker solution
			\end{itemize}
	\end{itemize}
	The government would also be a primary stakeholder as a military entity. So far in 2024, the US federal government has devoted 14\% of its budget to National Defense (about \$363 B). Part of this budget is devoted to surveillance and reconnaissance, and these days such operations are often performed via drone. 
	
	A road crossing in a foreign land presents a unique danger to our military. Such a crossing can both conceal and aid in the transport of enemy vehicles. To be able to identify such crossings from drone or satellite images would give the military a more informed lay of the land. This would allow the military to travel along less dangerous routes and would allow for more accurately targeted airstrikes. Furthermore, it's entirely possible that our model will accurately predict crossings where human eyes fail to do so. 
	
	Our crossing count will help achieve the following key performance indicators:
	
	\begin{itemize}
		\item Reduce military weapon usage and cost
		\begin{itemize}
			\item A missile can cost in the hundreds of thousands or millions of dollars
			\item A recon drone, in contrast, can be produced for as low as \$6000
		\end{itemize}
		\item Safer transportation routes for military vehicles
			\begin{itemize}
				\item Obtained by either avoiding crossings or sending drones to investigate crossings
			\end{itemize}
		\end{itemize}
		

	
\end{document}